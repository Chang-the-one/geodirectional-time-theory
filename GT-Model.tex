
\documentclass[12pt]{article}
\usepackage{amsmath,amssymb}
\usepackage{geometry}
\geometry{margin=1in}
\title{On the Non-Fundamentality of Time: A Directional Field Reformulation}
\author{Chang Liu}
\date{\today}

\begin{document}
\maketitle

\begin{abstract}
This paper proposes a directional field formulation of physical evolution without the assumption of fundamental time. Instead of relying on a time variable $t$, we define a unit direction tensor field $\mathcal{T}^\mu(x)$ that governs field propagation.
\end{abstract}

\section{Introduction}
Time is traditionally treated as a fundamental dimension...

\section{Core Equation}
We propose the following equation:
\[
\nabla_\mu \left( \mathcal{T}^\mu \mathcal{T}^\nu \nabla_\nu \Phi \right) = J(x)
\]

\section{Interpretation}
$\mathcal{T}^\mu(x)$ represents the local direction of physical evolution. Time becomes emergent...

\section{Consequences}
- Time is not needed as a coordinate
- Evolution proceeds via structural directionality

\section{Conclusion}
We provide a computable, direction-based alternative to time-based evolution in physics.

\end{document}
